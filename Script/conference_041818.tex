\documentclass[conference]{IEEEtran}
\IEEEoverridecommandlockouts
% The preceding line is only needed to identify funding in the first footnote. If that is unneeded, please comment it out.
\usepackage{cite}
\usepackage{amsmath,amssymb,amsfonts}
\usepackage{algorithmic}
\usepackage{graphicx}
\usepackage{textcomp}
\usepackage{xcolor}
\def\BibTeX{{\rm B\kern-.05em{\sc i\kern-.025em b}\kern-.08em
    T\kern-.1667em\lower.7ex\hbox{E}\kern-.125emX}}
\begin{document}

\title{LSTM based Beam Tracking for mmWave Vehicular Networks\\
}

\author{\IEEEauthorblockN{Chen Wang, XXX, XXX, XXX}
\IEEEauthorblockA{\textit{Beijing Key Laboratory of Network System Architecture and Convergence, Beijing University of Posts and Telecommunications} \\
Beijing, China \\
wangchen@bupt.edu.cn}
}

\maketitle

\begin{abstract}
The use of millimeter wave (mmWave) frequency bands for transmission can improve data rate with the help of beamforming technology to overcome the high path and penetration losses.
However for vehicle, the high mobility of vehicle results in extremely frequent beam alignment and significant overhead.
In this paper, a Long Short Time Memory (LSTM) based beam tracking method was proposed for reducing overhead brought by beam alignment in mmWave Vehicular Networks, by predicting angles of beam pair at next time step through known angles of beam pairs at a certain number of consecutive time steps as features.
To train this network, an time series array antenna channel data was set up by statistical channel model using time series vehicle information generated from road traffic simulation software named ``Simulation of Urban MObility (SUMO)''.
%% In addition, a extended Kalman filter (EKF) based method using vehicle information to estimate predicted beam angle was adopted as comparison method.
Simulation results show that proposed LSTM based method outpreforms Kalman filter based method, and can prevent frequent beam alignment to reduce overhead while ensuring acceptable signal-to-noise ratio (SNR).
\end{abstract}

\begin{IEEEkeywords}
beam tracking, mmWave vehicular networks, LSTM, Kalman filter
\end{IEEEkeywords}

\section{Introduction}
% 毫米波可以有效提高传输速率,是通信未来发展的重要技术 *。
% 为了应用更好的毫米波,需要克服毫米波高频导致的极高的路径和穿透损耗 *。
% 得益于毫米波毫米级的波长,可能将阵列天线集成进终端,并使用阵列天线的波束赋形技术,将天线能量约束在指定方向,获取高增益 *。
% 论文 * 提出了数字波束赋形技术,然而由于射频链的高价格和高能耗,给天线阵列中的每一根天线都配置射频链的成本是不可接受的 *。
% 目前较多的研究主要集中在基于量化相移器的单射频链模拟波束赋形 *,和使用较少射频链的混合波束赋形 * 折中方案上。

% 不同于传统全向天线的信道,阵列天线的信道具有方向性 *,这意味着若想形成较高的波束增益,必须保证接收端和发送端的波束之间相互对准,否则波束对的失准将会带来严重的增益下降,降低通信性能 *。
Unlike the channel of conventional omni-directional antennas, the channel of array antennas are directional *, which means that if higher antenna gain needs to be achieved, it is important to ensure that the beams at the receiver and transmitter are aligned with each other. Otherwise the misalignment of the beam pairs will result in serious gain reduction, degrading communications performance *.
% 研究 * 提出了一种分级码本的设计,本质是采用了二分查找的方式,在不同阶段使用不同宽度的波束进行波束扫描并获得最佳波束对,以此减少开支。
A hierarchical codebook design was proposed in * , which essentially use a dichotomous lookup to reduce overhead by using different widths of beams at different stages to perform beam search.
% 然而,这种算法需要量化相移器具有较高的量化水平,并且有可能需要多个射频链来实现,这导致采用这种算法方案的成本极高。
However, this algorithm requires a high quantization level phase shifter and potentially multiple RF chains to implement, which results in high cost.
% 对此,* 提出了一种基于卡尔曼滤波器使用穷尽扫描进行波束追踪的方案,以减少系统的开支,然而穷尽扫描所需要的测量次数和协议开支会随着天线数的增加而增加。
In \cite{Zhang2016}, Zhang et al. proposed a Kalman filter based method to 
% * 在此基础上加以改进采用 EKF,并且只需要单次测量,更适用于在快速变换环境下的信道追踪。

% 相较于低速的用户终端,为高速车辆提供可靠的波束对准更具挑战性,因为车辆较高的速度将会导致车辆更快驶出波束的覆盖范围。
Providing reliable beam alignment for high speed vehicles is more challenging than for low speed terminals because higher speeds cause vehicles to move out of the beam coverage faster. 
%为了保证车辆始终被较高增益的波束覆盖,需要进行高频率的对准,而高频的波束会进一步加大系统的开支,尤其是当采用高分辨率码本的时候。
To ensure that the vehicle is always covered by high gain beams, high frequency beam alignment is required, and it can further increase system overhead, especially when high resolution code books are used.
% 幸运的是,由于毫米波信道的稀疏性 *,且毫米波信道的 AoA、AoD 与收发端位置的具有一定的相关性 *,在 LOS 信号下沿固定道路行驶的车辆的 AoA 和 AoD 应该能被很好的预测出来。
Fortunately, due to the sparsity of mmWave channel * and the correlation between the AoA and AoD of a millimeter-wave channel and the location of the receiver and transmitter *, the channel information of a vehicle traveling along a fixed road under an LOS signal should be well predicted.

% 本文提出了一种基于 LSTM 网络的波束追踪方法,使用 LSTM 网络,通过对历史车辆穷尽扫描所获得的历史信道信息进行学习,根据当前车辆在一定时间步的信道信息,预测在下一时刻的 AoA 和 AoD,以达到减少系统开支的效果。
In this paper, a LSTM based beam tracking method is proposed. By using LSTM network to learn the historical channel information which was obtained from exhausted beam search, AoA and AoD of current channel between vehicle and BS can be predicted to reduce system overhead.
% 同时,为了获取信道数据用以训练神经网络,交通流仿真软件 SUMO 被用来生成时序车辆信息,结合统计信道模型与生成的时序车辆信息,生成了时序的信道信息。
Meanwhile, a road traffic simulation software is use to generate time seires vehicle information (e.g. position, speed, etc.) to set up time seires channel information with a statistical channel model for training the network.
%% 此外,提出了快速中断恢复算法,在追踪失效后能够减少波束扫描的次数,尽快恢复连接。
% 相比较于 *,本论文有以下几点不同:1)采用了更加完善的信道模型;2)同时预测 AoA 与 AoD;3)中断判断根据接收端的 SNR 而非角度差进行判断;4)在一个追踪周期中平均需要更少的波束测量。此外,与 * 相比使用了更实际的基于量化相移器的波束码本进行波束扫描。
Compared to \cite{Zhang2016, Va2017}, this paper differs in the following ways: 1) More realistic channel model; 2) Simultaneous prediction of AoA and AoD; 3) The outage judgment is based on the SNR of the receiver instead of the angle difference; 4) Fewer beam measurements required on average in a tracking cycle. In addition, beam search is performed more realistic than \cite{Va2017}, by using a quantized phase shifter based beam codebook.

% 内容安排

% 符号使用
The following notation will be used in this paper.
% 矩阵,向量,标量
Matrices, vectors and scalars are denoted by bold uppercase letters (e.g. $\mathbf{A}$), bold lowercase letters (e.g. $\mathbf{a}$) and lowercase letters (e.g. $a$), respectively.
% 转置,共轭转置
$(\cdot)^{T}$ denote transpose and  $(\cdot)^H$ denote conjugate transpose (Hermitian).
% 矩阵指定行、列、元素
$[\mathbf{A}]_{m,:}$, $[\mathbf{A}]_{:,n}$ and $[\mathbf{A}_{m,n}]$ denote the $m$th row, $n$th column and the $m$th row $n$th column entry of $\mathbf{A}$, respectively.
% 向量指定元素
$[mathbf{a}]_n$ denote the $n$th entry of $\mathbf{a}$.
% 2 范数
Beside, $\Vert{\cdot}\Vert_2$ denote $\ell_2$-norm of a vector.
% 复数、实数集合
$\mathbb{C}$ denote the set of complex number and $\mathbb{R}^+$ denote the set of real positive number.
% 复高斯、包络高斯、指数分布
Complex Gaussian distribution, wrapped Gaussian distribution and exponential distribution are denoted by $\mathcal{CN}$, $\mathcal{WN}$ and $\mathcal{E}$, respectively.

\section{System Model}

% 场景和天线数
A mmWave vehicular scenario including a vehicle with $M_r$ antenna as receiver and a Base Station (BS) with $M_t$ antenna as transmitter are considered. 
% 天线结构、阵元间距、赋形技术
Both of them equips with uniform linear array (ULA) of half wave interval antenna as shown in Fig.1, and adopted analog beamforming used quantitative phase shifter which connect with single analog radio frequency (RF) chain.
% 阵列响应矢量
The array response vector of a uniform linear array with $M$ half wave interval antenna is given by:
\begin{equation}
    \mathbf{a}( M,\theta ) =\frac{1}{\sqrt{M}}\left[ 1,e^{j\pi cos( \varphi )} ,...,e^{j\pi ( M-1) cos( \varphi )}\right]^{T}
    \end{equation}
where $\varphi$ is the arrival angle of the signal. So the array response vectors of the vehicle and BS are $\mathbf{a}_{r}( \phi ) =\mathbf{a}( M_{r} ,\phi )$ and $\mathbf{a}_{t}( \theta ) =\mathbf{a}( M_{t} ,\theta )$, respectively.

% 码本设计
The codebook matrix of vehicle is $\mathbf{W}$, and the codebook matrix of BS is $\mathbf{F}$. Each column of the codebook matrix represents a beam pattern, each entry in the column is phase rotation for corresponding antenna element to generate directional beam.
A discrete resolution $2log_2 M$-bit codebook is adopted, in other words, there are $2M_r$ beam patterns for the vehicle, and $M_t$ beam patterns for the BS.

% 信道模型
A statistical 28 GHz mmWave channel model in \cite{Akdeniz2014} is used, which is modeled by real experimental data collected in New York City.
The channel matrix for a $L$ subpaths at $n$th time step $\mathbf{H}_{n} \in \mathbb{C}^{M_{r} \times M_{t}}$ can be expressed as
\begin{equation}
    \mathbf{H}_{n} =\sum\limits ^{L}_{l=1} g_{ln}\mathbf{a}_{r}( \phi _{ln})\mathbf{a}^{H}_{t}( \theta _{ln})
    \end{equation}
where $g_{ln} \in \mathbb{C}$ is complex small-scale fading gain on subpath $l$ at $n$th time step, $\mathbf{a}_{r}( \phi _{ln}) \in \mathbb{C}^{M_r}$ and $\mathbf{a}_{t}( \theta _{ln}) \in \mathbb{C}^{M_t}$ are array response vectors of the vehicle and BS, respectively. $\phi_{ln}$ is AoA of the $l$th subpath signal received by the vehicle at $n$th time step, and $\theta_{ln}$ is AoD of the $l$th subpath signal transmitted by the BS at $n$th time step. The specific parameters can be refered to \cite{Akdeniz2014} and will not be repeated in this paper. It should be noted that in this paper, only the case of single cluster under LOS condition is considered. In addition, the cluster angle depends on the geometric position of the vehicle and base station.

\section{Beam Tracking Solution}

\subsection{Equations}
Number equations consecutively. To make your 
equations more compact, you may use the solidus (~/~), the exp function, or 
appropriate exponents. Italicize Roman symbols for quantities and variables, 
but not Greek symbols. Use a long dash rather than a hyphen for a minus 
sign. Punctuate equations with commas or periods when they are part of a 
sentence, as in:
\begin{equation}
a+b=\gamma\label{eq}
\end{equation}

Be sure that the 
symbols in your equation have been defined before or immediately following 
the equation. Use ``\eqref{eq}'', not ``Eq.~\eqref{eq}'' or ``equation \eqref{eq}'', except at 
the beginning of a sentence: ``Equation \eqref{eq} is . . .''

\subsection{\LaTeX-Specific Advice}

Please use ``soft'' (e.g., \verb|\eqref{Eq}|) cross references instead
of ``hard'' references (e.g., \verb|(1)|). That will make it possible
to combine sections, add equations, or change the order of figures or
citations without having to go through the file line by line.

Please don't use the \verb|{eqnarray}| equation environment. Use
\verb|{align}| or \verb|{IEEEeqnarray}| instead. The \verb|{eqnarray}|
environment leaves unsightly spaces around relation symbols.

Please note that the \verb|{subequations}| environment in {\LaTeX}
will increment the main equation counter even when there are no
equation numbers displayed. If you forget that, you might write an
article in which the equation numbers skip from (17) to (20), causing
the copy editors to wonder if you've discovered a new method of
counting.

{\BibTeX} does not work by magic. It doesn't get the bibliographic
data from thin air but from .bib files. If you use {\BibTeX} to produce a
bibliography you must send the .bib files. 

{\LaTeX} can't read your mind. If you assign the same label to a
subsubsection and a table, you might find that Table I has been cross
referenced as Table IV-B3. 

{\LaTeX} does not have precognitive abilities. If you put a
\verb|\label| command before the command that updates the counter it's
supposed to be using, the label will pick up the last counter to be
cross referenced instead. In particular, a \verb|\label| command
should not go before the caption of a figure or a table.

Do not use \verb|\nonumber| inside the \verb|{array}| environment. It
will not stop equation numbers inside \verb|{array}| (there won't be
any anyway) and it might stop a wanted equation number in the
surrounding equation.

\subsection{Some Common Mistakes}\label{SCM}
\begin{itemize}
\item The word ``data'' is plural, not singular.
\item The subscript for the permeability of vacuum $\mu_{0}$, and other common scientific constants, is zero with subscript formatting, not a lowercase letter ``o''.
\item In American English, commas, semicolons, periods, question and exclamation marks are located within quotation marks only when a complete thought or name is cited, such as a title or full quotation. When quotation marks are used, instead of a bold or italic typeface, to highlight a word or phrase, punctuation should appear outside of the quotation marks. A parenthetical phrase or statement at the end of a sentence is punctuated outside of the closing parenthesis (like this). (A parenthetical sentence is punctuated within the parentheses.)
\item A graph within a graph is an ``inset'', not an ``insert''. The word alternatively is preferred to the word ``alternately'' (unless you really mean something that alternates).
\item Do not use the word ``essentially'' to mean ``approximately'' or ``effectively''.
\item In your paper title, if the words ``that uses'' can accurately replace the word ``using'', capitalize the ``u''; if not, keep using lower-cased.
\item Be aware of the different meanings of the homophones ``affect'' and ``effect'', ``complement'' and ``compliment'', ``discreet'' and ``discrete'', ``principal'' and ``principle''.
\item Do not confuse ``imply'' and ``infer''.
\item The prefix ``non'' is not a word; it should be joined to the word it modifies, usually without a hyphen.
\item There is no period after the ``et'' in the Latin abbreviation ``et al.''.
\item The abbreviation ``i.e.'' means ``that is'', and the abbreviation ``e.g.'' means ``for example''.
\end{itemize}
An excellent style manual for science writers is \cite{b7}.

\subsection{Authors and Affiliations}
\textbf{The class file is designed for, but not limited to, six authors.} A 
minimum of one author is required for all conference articles. Author names 
should be listed starting from left to right and then moving down to the 
next line. This is the author sequence that will be used in future citations 
and by indexing services. Names should not be listed in columns nor group by 
affiliation. Please keep your affiliations as succinct as possible (for 
example, do not differentiate among departments of the same organization).

\subsection{Identify the Headings}
Headings, or heads, are organizational devices that guide the reader through 
your paper. There are two types: component heads and text heads.

Component heads identify the different components of your paper and are not 
topically subordinate to each other. Examples include Acknowledgments and 
References and, for these, the correct style to use is ``Heading 5''. Use 
``figure caption'' for your Figure captions, and ``table head'' for your 
table title. Run-in heads, such as ``Abstract'', will require you to apply a 
style (in this case, italic) in addition to the style provided by the drop 
down menu to differentiate the head from the text.

Text heads organize the topics on a relational, hierarchical basis. For 
example, the paper title is the primary text head because all subsequent 
material relates and elaborates on this one topic. If there are two or more 
sub-topics, the next level head (uppercase Roman numerals) should be used 
and, conversely, if there are not at least two sub-topics, then no subheads 
should be introduced.

\subsection{Figures and Tables}
\paragraph{Positioning Figures and Tables} Place figures and tables at the top and 
bottom of columns. Avoid placing them in the middle of columns. Large 
figures and tables may span across both columns. Figure captions should be 
below the figures; table heads should appear above the tables. Insert 
figures and tables after they are cited in the text. Use the abbreviation 
``Fig.~\ref{fig}'', even at the beginning of a sentence.

\begin{table}[htbp]
\caption{Table Type Styles}
\begin{center}
\begin{tabular}{|c|c|c|c|}
\hline
\textbf{Table}&\multicolumn{3}{|c|}{\textbf{Table Column Head}} \\
\cline{2-4} 
\textbf{Head} & \textbf{\textit{Table column subhead}}& \textbf{\textit{Subhead}}& \textbf{\textit{Subhead}} \\
\hline
copy& More table copy$^{\mathrm{a}}$& &  \\
\hline
\multicolumn{4}{l}{$^{\mathrm{a}}$Sample of a Table footnote.}
\end{tabular}
\label{tab1}
\end{center}
\end{table}

\begin{figure}[htbp]
% \centerline{\includegraphics{}}
\caption{Example of a figure caption.}
\label{fig}
\end{figure}

Figure Labels: Use 8 point Times New Roman for Figure labels. Use words 
rather than symbols or abbreviations when writing Figure axis labels to 
avoid confusing the reader. As an example, write the quantity 
``Magnetization'', or ``Magnetization, M'', not just ``M''. If including 
units in the label, present them within parentheses. Do not label axes only 
with units. In the example, write ``Magnetization (A/m)'' or ``Magnetization 
\{A[m(1)]\}'', not just ``A/m''. Do not label axes with a ratio of 
quantities and units. For example, write ``Temperature (K)'', not 
``Temperature/K''.

\section*{Acknowledgment}

The preferred spelling of the word ``acknowledgment'' in America is without 
an ``e'' after the ``g''. Avoid the stilted expression ``one of us (R. B. 
G.) thanks $\ldots$''. Instead, try ``R. B. G. thanks$\ldots$''. Put sponsor 
acknowledgments in the unnumbered footnote on the first page.

\section*{References}

Please number citations consecutively within brackets \cite{b1}. The 
sentence punctuation follows the bracket \cite{b2}. Refer simply to the reference 
number, as in \cite{b3}---do not use ``Ref. \cite{b3}'' or ``reference \cite{b3}'' except at 
the beginning of a sentence: ``Reference \cite{b3} was the first $\ldots$''

Number footnotes separately in superscripts. Place the actual footnote at 
the bottom of the column in which it was cited. Do not put footnotes in the 
abstract or reference list. Use letters for table footnotes.

Unless there are six authors or more give all authors' names; do not use 
``et al.''. Papers that have not been published, even if they have been 
submitted for publication, should be cited as ``unpublished'' \cite{b4}. Papers 
that have been accepted for publication should be cited as ``in press'' \cite{b5}. 
Capitalize only the first word in a paper title, except for proper nouns and 
element symbols.

For papers published in translation journals, please give the English 
citation first, followed by the original foreign-language citation \cite{b7}.

\bibliographystyle{IEEEtran}
\bibliography{references}

% \begin{thebibliography}{00}
% \bibitem{b1}
% \end{thebibliography}

\end{document}
